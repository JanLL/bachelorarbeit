\documentclass{beamer}

\usepackage[ngerman]{babel}
\usepackage[utf8x]{inputenc}
\usepackage{amsmath,amsfonts,amssymb}
%\usepackage[dvipsnames]{xcolor}

\definecolor{GTarea}{RGB}{229, 165, 212}
\definecolor{GTline}{RGB}{107, 0, 78}
\definecolor{structure}{RGB}{166, 166, 166}



\usepackage[absolute,overlay]{textpos}


\usetheme{Luebeck}
\usecolortheme{orchid}

\title{Bachelor Thesis}
%\subtitle{Vergleich von Hamming- und Variation of Information-Loss basiertem strukturiertem Parameterlernen beim Multicut Problem}
\subtitle{Comparison of Hamming- and Variation of Information-Loss based structured learning on the Multicut Problem}
\author{Jan Lammel}

\begin{document}
	
\frame{\titlepage}

\frame{
	\frametitle{Table of Contents}
	\tableofcontents[hideallsubsections]
	
}

\section{Introduction}

\subsection{Problem}

\frame{
	\frametitle{Segmentation}
	\begin{textblock}{5}(1, 4.5)
		\includegraphics[width=1.0\textwidth]{images/80090.png}
	\end{textblock}
	
	\begin{textblock}{2.5}(6.3, 8)
		\includegraphics[width=1.0\textwidth]{images/pfeil.png}
	\end{textblock}
	
	\begin{textblock}{5}(9, 4.5)
		\includegraphics[width=1.0\textwidth]{images/80090_gtg.png}
	\end{textblock}	
	
	
}
	
\frame{
	\frametitle{Motivation Variation of Information}
	\begin{textblock}{5}(1, 4.5)
		\includegraphics[width=1.0\textwidth]{images/159022.jpg}
	\end{textblock}
	\begin{textblock}{5}(1, 10)
		\includegraphics[width=1.0\textwidth]{images/motivation.png}
	\end{textblock}	

	\begin{textblock}{9}(6.5, 4.5)
		\begin{itemize}
			\item Hamming Loss strongly dependend on exact path of segmentation
			\item But: Path of segmentation often not unique
			\item Idea VOI: Consider labels of segmentation and penanlize area-dependend
		\end{itemize}
	\end{textblock}
}

\section{Theory}
\subsection{Region Adjacency Graph}
\frame{
	\frametitle{Region Adjacency Graph (RAG)}
	\begin{textblock}{4}(1, 4.5)
		\includegraphics[width=1.0\textwidth]{images/rag2rot.png}
	\end{textblock}
	
	\begin{textblock}{9}(6.5, 5.5)
		\begin{itemize}
			\item Image partitioned into \textcolor{red}{Superpixel} (SP) via SLIC
			\item Each Superpixel $\hat{=}$ \textcolor[rgb]{0.1333,0.576,0.165}{Node} in RAG
			\item Nodes of adjacent SP are linked by an \textcolor{blue}{Edge}
		\end{itemize}
	\end{textblock}
}


\subsection{Multicut Problem}
\frame{
	\frametitle{Multicut Problem (MP)}
	\begin{textblock}{16}(0,5)
		
		\begin{equation*} 
		\begin{array}{rrclcl}
		\displaystyle \min_{y} & \sum\limits_{y_e \in E} \langle w, \beta_e \rangle \cdot y_e \\
		\textrm{s.t.} &  y - \sum\limits_{y_i \in P(y)} y_i & \le & 0 & & \forall \ y \in E
		\end{array}
		\label{eq:mp}
		\end{equation*}
	\end{textblock}
	
	\begin{textblock}{12}(1,9)
		\begin{itemize}
			\item $w$: Weights to be learned
			\item $\beta_e$: Features of edge $e$
			\item $y_e$: Activity of edge $e$
			\item Constraint to enforce consistency
		\end{itemize}
	\end{textblock}
}


\subsection{Learning}
\frame{
	



}


	
\section{Experimental Setup}
\subsection{Training- and Test data}
\frame{
	\frametitle{Training- and Test data}
	\begin{textblock}{14}(1,5)
		\begin{itemize}
			\item Natural Images from Berkeley Segmentation Dataset (BSD-500)
			\item Took thereof the 200 images from the testset due to availability of state-of-the-art contour detectors
			$\rightarrow$ 100 training- and test- images
			\item Ground Truth as well from BSD-500 dataset (determined label of SP via majority vote)
		\end{itemize}
	\end{textblock}

}

\subsection{Feature Space}
\frame{
	\frametitle{Feature Space}
	\begin{textblock}{14}(1,5)
		\begin{itemize}
			\item Gaussian Gradient Magnitude
			\item Hessian of Gaussian Eigenvalues
			\item Laplacian of Gaussian
			\item Structure Tensor Eigenvalues
			\item Canny Filter
			\item $N^4$-Fields \cite{!!!} with and without edge length weighting
			\item Dollár et. al \cite{!!!} Kantendetektor with and without edge length weighting
		\end{itemize}
	\end{textblock}	
}

\frame{
	\frametitle{Feature Space}
	\begin{textblock}{14}(1,5)
		\begin{itemize}
			\item Statistics in area $\tilde{u}$ und $\tilde{v}$ around edge of SP u and v:
			\begin{itemize}
				\item Mean($\tilde{u} + \tilde{v}$)
				\item Variance($\tilde{u} + \tilde{v}$)
				\item $\frac{\max{\{\text{Mean}(\tilde{u}), \text{Mean}(\tilde{v}) \}}}{\min{\{\text{Mean}(\tilde{u}), \text{Mean}(\tilde{v})}\}}$
				\item $\frac{\max{\{\text{Median}(\tilde{u}), \text{Median}(\tilde{v}) \}}}{\min{\{\text{Median}(\tilde{u}), \text{Median}(\tilde{v})}\}}$
				\item Skewness($\tilde{u} + \tilde{v}$)
				\item Kurtosis($\tilde{u} + \tilde{v}$)
			\end{itemize}
			\item Random Forest Feature
		\end{itemize}
	\end{textblock}	
	
	\begin{textblock}{6}(8,6.7)
		\includegraphics[width=1.0\textwidth]{images/bereich-um-sp.png}
	\end{textblock}

}




\section{Experiments and Results}
\subsection{Stochastic Gradient with RF Feature}
\frame{
	\frametitle{Stochastic Gradient with RF Feature}
	\begin{textblock}{14}(1,5)
		\begin{itemize}
			\item Varying configurations:
			\begin{itemize}
				\item Domain Feature Space
				\item Constraint on RF Feature
				\item Subgradient Descent with/without RF Feature 
			\end{itemize}
		
			\item Results:
			\begin{itemize}
				\item Decrease of VOI Loss leads to decline of Hamming Loss in Trainingsset
				\item Rate of decrease sensible to configuration, \\
				besides strong fluctuations due to stochastic process
				\item Loss decrease on Trainingset $\propto$ Loss increase on Testset \\
				$\Rightarrow$ Overfit of training data
				
			\end{itemize}
		\end{itemize}
	\end{textblock}	

}

\subsection{Stochastic Gradient without RF Feature}
\frame{
	\frametitle{Stochastic Gradient without RF Feature}
	\begin{textblock}{14}(1,5)
		\begin{itemize}
			\item Varying configurations:
			\begin{itemize}
				\item Domain Feature Space
				\item Constraint on $N^4$ Feature
			\end{itemize}
		
			\item Results:
			\begin{itemize}
				\item VOI Loss decrease on Trainingset of approximately 4\%
				\item Change of Loss on Testset within $1 \sigma$ range of error
			\end{itemize}
		\end{itemize}
	\end{textblock}	
}

\subsection{Cross Validation Measurement 10}
\frame{
	\frametitle{Cross Validation Measurement 10}
	\begin{textblock}{14}(1,4.5)
		\begin{itemize}
			\item Cross validation to minimize measurement errors
		
			\item Results on Testset:
			\begin{itemize}
				\item $\mathcal{L}_{H}$: $\frac{StochGrad}{SubGrad} = 0.989 \pm 0.005$ \\
				\vspace{0.1cm} \item $\mathcal{L}_{VOI}$: $\frac{StochGrad}{SubGrad} = 1.0025 \pm 0.0084$ \\
				\vspace{0.2cm} $\rightarrow$ No significant change
			\end{itemize}
		\end{itemize}
	\end{textblock}	
}

\frame{
	\frametitle{Explanation by SLIC}
	\begin{textblock}{7}(2,5.5)
		\includegraphics[width=1.0\textwidth]{images/slic-gt1.png}
	\end{textblock}	

	\begin{textblock}{1}(10, 6)
		\fboxsep=2mm %\fboxrule=1mm
		\fcolorbox{black}{GTline}{\null} \\
		\fcolorbox{black}{GTarea}{\null} \\
		\fcolorbox{black}{gray}{\null}  \\			
	\end{textblock}
	
	\begin{textblock}{7}(10.7,6.1)
		Ground Truth Edge \\
		Ground Truth Area \\
		Structure \\
	\end{textblock}

}


\frame{
	\frametitle{Explanation by SLIC}
	\begin{textblock}{7}(2,5.5)
		\includegraphics[width=1.0\textwidth]{images/slic-gt2.png}
	\end{textblock}	

	\begin{textblock}{1}(10, 6)
		\fboxsep=2mm %\fboxrule=1mm
		\fcolorbox{black}{GTline}{\null} \\
		\fcolorbox{black}{GTarea}{\null} \\
		\fcolorbox{black}{gray}{\null}  \\	
		\fcolorbox{black}{red}{\null}  \\	
						
	\end{textblock}
	
	\begin{textblock}{7}(10.7,6.1)
		Ground Truth Edge \\
		Ground Truth Area \\
		Structure \\
		Super Pixel Edges \\
	\end{textblock}

}

\frame{
	\frametitle{Explanation by SLIC}
	\begin{textblock}{7}(2,5.5)
		\includegraphics[width=1.0\textwidth]{images/slic-gt3.png}
	\end{textblock}	

	\begin{textblock}{1}(10, 6)
		\fboxsep=2mm %\fboxrule=1mm

		\fcolorbox{black}{gray}{\null}  \\	
		\fcolorbox{black}{red}{\null}  \\	
						
	\end{textblock}
	
	\begin{textblock}{7}(10.7,6.1)
		Structure \\
		Super Pixel- \& \\ Ground Truth Edges \\
	\end{textblock}

}


\section{Conclusion}
\frame{
	\frametitle{Conclusion}

	
	\begin{textblock}{14}(1., 5)
		\begin{itemize}
			\item Stochastic Gradient with RF Feature leads to Overfit of training data
			\item No significant change without RF Feature
			\begin{itemize}
				\item Bad Ground Truth compensated by SLIC
				\item Examination on Pixel-level would be interesting
			\end{itemize}
			\item Examples were found which confirm theoretical behavior of VOI
		\end{itemize}
	\end{textblock}


}

	
	
	
\end{document}