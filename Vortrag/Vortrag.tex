\documentclass{beamer}

\usepackage[ngerman]{babel}
\usepackage[utf8x]{inputenc}
\usepackage{amsmath,amsfonts,amssymb}

\usepackage[absolute,overlay]{textpos}


\usetheme{Luebeck}
\usecolortheme{orchid}

\title{Bachelor Thesis}
\subtitle{Vergleich von Hamming- und Variation of Information-Loss basiertem strukturiertem Parameterlernen beim Multicut Problem}
\author{Jan Lammel}

\begin{document}
	
\frame{\titlepage}

\frame{
	\frametitle{Inhaltsverzeichnis}
	\tableofcontents[hideallsubsections]
	
}

\section{Einleitung}

\subsection{Problemstellung}

\frame{
	\frametitle{Segmentierung}
	\begin{textblock}{5}(1, 4.5)
		\includegraphics[width=1.0\textwidth]{images/80090.png}
	\end{textblock}
	\begin{textblock}{5}(9, 4.5)
		\includegraphics[width=1.0\textwidth]{images/80090_gtg.png}
	\end{textblock}	
}
	
\frame{
	\frametitle{Motivation Variation of Information}
	\begin{textblock}{5}(1, 4.5)
		\includegraphics[width=1.0\textwidth]{images/159022.jpg}
	\end{textblock}
	\begin{textblock}{5}(1, 10)
		\includegraphics[width=1.0\textwidth]{images/motivation.png}
	\end{textblock}	

	\begin{textblock}{9}(6.5, 4.5)
		\begin{itemize}
			\item Hamming Loss stark abhaengig vom genauen Verlauf der Segmentierung
			\item Aber: Verlauf der Segmentierung oft nicht eindeutig
			\item Idee VOI: Betrachte Label der Segmentierung und bestrafe flaechenabhaengig 
		\end{itemize}
	\end{textblock}
}

\section{Theoretische Grundlagen}
\subsection{Region Adjacency Graph}
\frame{
	\frametitle{Region Adjacency Graph (RAG)}
	\begin{textblock}{4}(1, 4.5)
		\includegraphics[width=1.0\textwidth]{images/rag2rot.png}
	\end{textblock}
	
	\begin{textblock}{9}(6.5, 5.5)
		\begin{itemize}
			\item Bild mittels SLIC in \textcolor{red}{Superpixel} (SP) unterteilt
			\item Jeder Superpixel $\hat{=}$ \textcolor[rgb]{0.1333,0.576,0.165}{Knoten} im RAG
			\item Direkt benachbarte SP erzeugen \textcolor{blue}{Kanten} zwischen entsprechenden Knoten
		\end{itemize}
	\end{textblock}
}


\subsection{Multicut Problem}
\frame{
	\frametitle{Multicut Problem (MP)}
	\begin{textblock}{16}(0,5)
		
		\begin{equation*} 
		\begin{array}{rrclcl}
		\displaystyle \min_{y} & \sum\limits_{y_i \in E} \langle w, \beta_e \rangle \cdot y_i \\
		\textrm{s.t.} &  y - \sum\limits_{y_i \in P(y)} y_i & \le & 0 & & \forall \ y \in E
		\end{array}
		\label{eq:mp}
		\end{equation*}
	\end{textblock}
}

\subsection{Loss Funktionen}
\frame{to-do}

\subsection{Structured Learning}
\frame{to-do}
	
\section{Experimentelles Setup}
\subsection{Trainings- und Testdaten}
\frame{
	\frametitle{Trainings- und Testdaten}
	\begin{textblock}{14}(1,5)
		\begin{itemize}
			\item Natuerliche Bilder vom Berkeley Segmentation Dataset (BSD-500) \\
			\item Davon die 200 Bilder des Testsets verwendet da hierfuer state-of-the-art Kantendetektoren zur Verfuegung standen \\
			$\rightarrow$ je 100 Trainings- und Testbilder
			\item Ground Truth ebenso aus BSD-500 Datensatz (mittels Majority Vote Label der SP bestimmt)
		\end{itemize}
	\end{textblock}

}

\subsection{Feature Space}
\frame{
	\frametitle{Feature Space}
	\begin{textblock}{14}(1,5)
		\begin{itemize}
			\item Gaussian Gradient Magnitude
			\item Hessian of Gaussian Eigenwerte
			\item Laplacian of Gaussian
			\item Structure Tensor Eigenwerte
			\item Canny Filter
			\item $N^4$-Fields \cite{!!!} mit und ohne Gewichtung der Kantenlaenge
			\item Dollár et. al \cite{!!!} Kantendetektor mit und ohne Gewichtung der Kantenlaenge
		\end{itemize}
	\end{textblock}	
}

\frame{
	\frametitle{Feature Space}
	\begin{textblock}{14}(1,5)
		\begin{itemize}
			\item Statistische Kenndaten in variablen Bereichen $\tilde{u}$ und $\tilde{v}$ um Kante der SP u und v:
			\begin{itemize}
				\item Mean($\tilde{u} + \tilde{v}$)
				\item Variance($\tilde{u} + \tilde{v}$)
				\item $\frac{\max{\{\text{Mean}(\tilde{u}), \text{Mean}(\tilde{v}) \}}}{\min{\{\text{Mean}(\tilde{u}), \text{Mean}(\tilde{v})}\}}$
				\item $\frac{\max{\{\text{Median}(\tilde{u}), \text{Median}(\tilde{v}) \}}}{\min{\{\text{Median}(\tilde{u}), \text{Median}(\tilde{v})}\}}$
				\item Skewness($\tilde{u} + \tilde{v}$)
				\item Kurtosis($\tilde{u} + \tilde{v}$)
			\end{itemize}
			\item Konstantes Feature zur Beseitigung des Bias der Decision Boundary
			\item Random Forest Feature aus den bisher genannten
		\end{itemize}
	\end{textblock}	
	
	\begin{textblock}{6}(8,6.7)
		\includegraphics[width=1.0\textwidth]{images/bereich-um-sp.png}
	\end{textblock}

}




\section{Experimente und Resultate}
\subsection{Stochastic Gradient mit RF Feature}
\frame{
	\frametitle{Stochastic Gradient mit RF Feature}
	\begin{textblock}{14}(1,5)
		\begin{itemize}
			\item Einstellungen variiert:
			\begin{itemize}
				\item Definitionsbereich Feature Space
				\item Constraint auf RF Feature
				\item Subgradient Descent mit/ohne RF Feature 
			\end{itemize}
		
			\item Ergebnisse:
			\begin{itemize}
				\item Verringerung des VOI Loss fuehrt auch auf Abfall des Hamming Loss im Trainingsset
				\item Geschwindigkeit des Abfalls stark von Konfiguration abhängig, \\
				ausserdem starke Schwankungen da stochastischer Prozess
				\item Verringerung Loss auf Trainingsset $\propto$ Erhoehung Loss auf Testset \\
				$\Rightarrow$ Overfit der Trainingsdaten
				
			\end{itemize}
		\end{itemize}
	\end{textblock}	

}

\subsection{Stochastic Gradient ohne RF Feature}
\frame{

}

\subsection{Kreuzvalidierung Messung 10}
\frame{

}


\section{Fazit}
\frame{Fazit}

	
	
	
\end{document}