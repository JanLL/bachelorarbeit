%*****************************************
\chapter{Theoretische Grundladen}\label{ch:theoretischeGrundlagen}
%*****************************************
\section{Graphen Theorie}\label{sec:graphTheory}

Die Grundlage aller weiteren Betrachtungen ist ein sogenannter Region Adjacency Graph (RAG). Dieser wird erstellt indem das Bild in ähnlich große Segmente (Superpixel genannt) unterteilt wird, dessen Ränder möglichst an Objektkonturen verlaufen. Dies ermöglicht der sogenannte SLIC-Algorithmus (Zitat!!), das Ergebnis hiervon ist in (Abb verlinken) abgebildet. \\

Der Region Adjacency Graph besteht aus sogenannten Nodes und Verbindungen zwischen diesen, Edges genannt. In unserem Fall entsprechen die Nodes den Superpixeln. Edges bestehen nur zwischen denjenigen Nodes, bei denen die zugehörigen Superpixel direkt angrenzen und eine gemeinsame Kante besitzen. 

(hier SLIC-partition Bild und RAG-Bild einfügen)





\section{Das Multicut Problem (MP)}\label{sec:multicutProb}







%*****************************************
%*****************************************
%*****************************************
%*****************************************
%*****************************************
