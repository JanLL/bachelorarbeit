%************************************************
\chapter{Experimentelles Setup}\label{ch:mathtest} % $\mathbb{ZNR}$
%************************************************

\section{Trainings- und Testdaten}

Zum einen wurden Experimente an kleinen synthetisch erzeugten Bildern durchgeführt um die prinzipiellen Vorteile von Variation of Information zu demonstrieren. 

Zum Anderen diente für die umfassenderen Experimente zur praktischen Anwendung das Berkeley Segmentation Dataset (BSD-500) \cite{BSD}, welches aus natürlichen Bildern besteht. Es wurden hiervon die Test-Bilder genommen, da hierfür State of the Art Kantendetektoren als Feature zur Verfügung standen. Sowohl das Trainings- als auch das Testset bestand aus 100 Bildern.


\section{Graphical Model Unterbau und Solver}


Zur Generierung des Region Adjacency Graphs, des Random Forests und der Filter wurde VIGRA \cite{VIGRA} verwendet. Inferno \cite{Inferno} zum Zusammenführen aller Daten, Lösen des Multicut Problems und Lernen der Parameter sowohl mit SubGradient bezüglich Partition Hamming, als auch mit Stochastic Gradient bezüglich Variation of Information. Beide Bibliotheken basieren auf C++, welche allerdings über Python angesteuert werden können. Daher wurde das komplette Programm für diese Arbeit in Python realisiert.



\section{Feature Space}

Für die synthetischen Daten wurden folgende Feature gewählt:
\begin{itemize}
	\item Gaussian Gradient Magnitude mit $\sigma=1$
	\item Hessian of Gaussian Eigenvalues mit $\sigma=1$
	\item Structure Tensor Eigenvalues
\end{itemize}

Beim BSD hat sich der Feature Space folgendermaßen zusammengesetzt:

\begin{itemize}
	\item Gaussian Gradient Magnitude mit $\sigma=\{1, 2, 5\}$
	\item Hessian of Gaussian Eigenvalues mit $\sigma=2$
	\item Laplacian of Gaussian
	\item Structure Tensor Eigenvalues
	\item Canny Filter
	\item $N^4$-Fields Kantendetektor \cite{n4}
	\item Structured Forests Kantendetektor \cite[Dollár et al.]{dollar}
	\item Statistische Kenndaten in variablen Bereichen $\bar{u}$ und $\bar{v}$ um eine Edge an Superpixeln u und v \\
	(seperat angewandt auf alle 3 Farbkanäle des eigentlichen Bildes, als auch auf den $N^4$-Fields- und Dollár-Kantendetektor)
	\begin{itemize}
		\item Mean($\bar{u} + \bar{v}$)
		\item Variance($\bar{u} + \bar{v}$)
		\item $\frac{\max{\{\text{Mean}(\bar{u}), \text{Mean}(\bar{v}) \}}}{\min{\{\text{Mean}(\bar{u}), \text{Mean}(\bar{v})}\}}$
		\item $\frac{\max{\{\text{Median}(\bar{u}), \text{Median}(\bar{v}) \}}}{\min{\{\text{Median}(\bar{u}), \text{Median}(\bar{v})}\}}$
		\item Skewness($\bar{u} + \bar{v}$)
		\item Kurtosis($\bar{u} + \bar{v}$)
	\end{itemize}
	\item Konstantes Feature für jede Edge, zur Beseitigung des Bias im Feature Space
\end{itemize}

Zusätzlich wurde aus den Feature Spaces aller Trainingsdaten ein Random Forest (RF) aufgebaut und dieser zur Generierung eines weiteren Features verwendet.







%*****************************************
%*****************************************
%*****************************************
%*****************************************
%*****************************************
