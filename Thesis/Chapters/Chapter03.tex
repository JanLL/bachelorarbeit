%************************************************
\chapter{Experimentelles Setup}\label{ch:mathtest} % $\mathbb{ZNR}$
%************************************************

\section{Trainings- und Testdaten}

Zum einen wurden Experimente an kleinen synthetisch erzeugten Bildern durchgeführt um die prinzipiellen Vorteile von Variation of Information zu demonstrieren. 

Zum Anderen diente für die umfassenderen Experimente zur praktischen Anwendung das Berkeley Segmentation Dataset (BSD500) \cite{BSD}, welches aus natürlichen Bildern besteht. Es wurden hiervon die Test-Bilder genommen, da hierfür State of the Art Kantendetektoren als Feature zur Verfügung standen.


\section{Feature Space}

Für die synthetischen Daten wurden folgende Feature gewählt:
\begin{itemize}
	\item Gaussian Gradient Magnitude mit $\sigma=1$
	\item Hessian of Gaussian Eigenvalues mit $\sigma=1$
	\item Structure Tensor Eigenvalues
\end{itemize}

Beim BSD hat sich der Feature Space folgendermaßen zusammengesetzt:

\begin{itemize}
	\item Gaussian Gradient Magnitude mit $\sigma=\{1, 2, 5\}$
	\item Hessian of Gaussian Eigenvalues mit $\sigma=2$
	\item Laplacian of Gaussian
	\item Structure Tensor Eigenvalues
	\item Canny Filter
	\item Statistische Kenndaten in variablen Bereichen $\bar{u}$ und $\bar{v}$ um eine Edge an Superpixeln u und v
	\begin{itemize}
		\item Mean($\bar{u} + \bar{v}$)
		\item Variance($\bar{u} + \bar{v}$)
		\item $\frac{\max{\{\text{Mean}(\bar{u}), \text{Mean}(\bar{v}) \}}}{\min{\{\text{Mean}(\bar{u}), \text{Mean}(\bar{v})}\}}$
		\item $\frac{\max{\{\text{Median}(\bar{u}), \text{Median}(\bar{v}) \}}}{\min{\{\text{Median}(\bar{u}), \text{Median}(\bar{v})}\}}$
		\item Skewness($\bar{u} + \bar{v}$)
		\item Kurtosis($\bar{u} + \bar{v}$)
	\end{itemize}
	\item 
\end{itemize}





%*****************************************
%*****************************************
%*****************************************
%*****************************************
%*****************************************
